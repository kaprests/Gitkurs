\documentclass{beamer}

\title{Git-kurs}
\subtitle{Versjonkontroll med WebKom}

\usetheme{metropolis}
%\usecolortheme{seagull}

\begin{document}
    \frame{
        \titlepage
    }

    \frame{
        \frametitle{Hva er git?}
        \begin{block}{GitHub}
            \begin{center}
                \huge Git $\neq$ GitHub!
            \end{center}    
        \end{block} 

        \begin{itemize}
            \item Git
                \begin{itemize}
                    \item versjonkontrollsprogram
                \end{itemize}
            \item GitHub
                \begin{itemize}
                    \item Vertstjeneste for git-repoer
                    \item Finnes flere alternativer (GitLab, Bitbucket etc.)
                \end{itemize}
        \end{itemize}
    }

    \frame{
        \frametitle{Terminologi}
        Noen viktige begreper
        \begin{itemize}
            \item Commit
            \item Branch
            \item Repository
                \begin{itemize}
                    \item lokalt/remote
                \end{itemize}
        \end{itemize}
        Om en forstår disse begrepene forstår en git
    }

    \frame{
        \frametitle{Commit}
        Et slags "snapshot" av koden/prosjektet
        \\
        En commit består av:
        \begin{itemize}
            \item Endring fra forrige commit
            \item forelder
            \item hashkode
            \item metadata
        \end{itemize}
        En oppretter commits på logiske/taktiske tidpunkter, f.eks. etter å ha fikset en bug 
        eller implementert en ny funksjon.
        \\-\\
        En kan alltid returnere til en tidligere commit
    }

    \frame{
        \frametitle{Branch}
        Alle commits tilhører en branch\\
        Hovedbranch: master \\
        Head: referanse til nåverende branch\\
        Merging
        \begin{itemize}
            \item Sammenslåing av branches
            \item Mergekonflikter
        \end{itemize}
        For større endringer (mer enn en commit) oppretter en gjerne en egen branch som man commiter endringene 
        på, før en merger den inn i master.
    }

    \frame{
        \frametitle{Repository}
        Som regel prosjektmappen\\
        Inneholder:
        \begin{itemize}
            \item Alle filene du vil holde styr på og deres historikk
            \item Alle commits
        \end{itemize}
        Kan lagres på f.eks. GitHub
        \begin{itemize}
            \item cloning
            \item pull/push 
        \end{itemize}
    }

    \frame{
        \frametitle{Hvordan bruke git}
        Det finnes GUI
        \begin{itemize}
            \item ikke standard
        \end{itemize}
        Standard å bruke kommandolinjen\\
        Ulike IDE-er har ofte git-integrasjon og mulighet for å gjøre ting med GUI
    }

    \frame{
        \frametitle{Kommandoer}
        Oversikt over de viktigste kommandoene
        \begin{itemize}
            \item git init
            \item git add
            \item git commit
            \item git pull
            \item git push
            \item git branch
            \item git checkout
            \item git merge
            \item git clone
        \end{itemize}
    }

    \frame{
        \frametitle{Git arbeidsflyt - lokalt repo på din pc}
        \begin{itemize}
            \item Gjør endringer
            \item stage endringer med git add
            \item lag commit med git commit
        \end{itemize}
    }

    \frame{
        \frametitle{Git arbeidsflyt - repo med kun deg, med remote}
        \begin{itemize}
            \item Gjør endringer
            \item stage endringer med git add
            \item lag commit med git commit
            \item push endringer til origin (GitHub) med git push
        \end{itemize}
    }

    \frame{
        \frametitle{Git arbeidsflyt - remote med flere bidragsytere, liten endring}
        \begin{itemize}
            \item Gjør endringer
            \item stage endringer med git add
            \item lag commit med git commit
            \item hent nyeste versjon fra origin med git pull
            \item push endringer til origin (GitHub) med git push
        \end{itemize}
    }

    \frame{
        \frametitle{Git arbeidsflyt - remote med flere bidragsytere, større endring}
        \begin{itemize}
            \item Lag ny branch med git branch \textless navn på ny branch\textgreater
            \item Bytt til din nye branch med git checkout \textless navn på branch\textgreater
            \item Gjenta til endringen er klar
                \begin{itemize}
                    \item Gjør endringer
                    \item stage endringer med git add
                    \item lag commit med git commit
                \end{itemize}
            \item Bytt til master branch med git checkout master
            \item Hent nyeste versjon fra origin med git pull
            \item Merge din branch med endringer inn i master med git merge \textless navn på din branch\textgreater
            \item push endringer til origin (GitHub) med git push
        \end{itemize}
    }

    \frame{
        \frametitle{Oppgaver}
    }
\end{document}


